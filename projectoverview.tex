\section{Overview}
This part covers the project that has been developed for this thesis. It contains a Web application that has been built twice with two completely different architectural approaches. The main goal in this thesis is to find a superior software architecture for a typical Web 2.0 application, by comparing a traditional architectural approach with a modern and innovative approach. In order to evaluate these two architectures, the writer of this thesis has invented a Web app that represents a traditional Web 2.0 application. This Web app contains the most commonly seen features of a traditional web 2.0 application. This includes:
\begin{itemize}
\item{} Social networking interactions:
	\begin{itemize}
		\item{} Have a user-profile that is publicly visible to other users
		\item{} Connect to other users, for example in a friendship relationship
		\item{} Creating blogs and upload posts to it 
		\item{} Ability to comment and rate blog posts
	\end{itemize}
\item{} Interactive behavior with rich user interfaces
\item{} Large amounts of persisted data (this mainly because the app-users themselves create the information content)
\end{itemize}

The first architecture that was built conforms to a traditional approach, and the second conforms to a modern approach. 

The rest of this part is separated into three chapters: In the first chapter we will look at the Web app itself, seen from the end-user's perspective. A description of the Web app's user behavior and requirements is necessary in order to understand the solutions that was taken when designing the software architecture for the app. The Web app is named Shredhub. In the last two chapters, we will have a detailed look at the two ways the app was built. The architecture outlined in the second chapter, is named Architecture 1.0, and is adheres to the principles from \textit{reference-model 1.0}. The final chapter discusses the architectural details in Architecture 2.0, which is based on \textit{reference-model 2.0}. For the record, during the discussion we will use the term \textit{User} to refer to the currently logged in user.