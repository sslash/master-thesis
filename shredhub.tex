\chapter{Shredhub, a Web 2.0 Application}
	
\section {Shredhub}
Shredhub, is a social web application for musicians, aimed primarily for guitarists. The application enables users to share their skills and musical passion in a social and competing manner. Through a modern and interactive user interface, the users are able to post videos of themselves playing a short tune. Everyone can watch, comment and give a numbered rating to the videos, so that the creator can achieve experience points and become highly ranked on this social platform. The purpose of Shredhub is to gather guitar players from across the world to create a community of competent and enthusiastic musicians. Having a social network of people who share the same interest is a great way for people to both learn, and have fun with their playing. People can deploy their musical ideas, show the world how beautiful their guitar sounds, challenge a friend or stranger in a battle, or whatever would suit their needs. 
		
Tt is important to acknowledge the fact that this application is primarily for guitarists, which does imply a slightly small user group. A better solution would be to implement a system that supported more kinds of musicians, for instance drummers, piano players, saxophone players etc. Therefore, a better solution could be to let the user pick their preferred instrument before they access the application's main page. From there on they would only be able to participate with the kinds of musicians the user picked at startup. However, because I have only had a certain amount of time to implement application, extending the application unfortunately goes out of project scope. Therefore I content myself with only supporting guitar players in this project. 
		
\section {User functionality}
The term \textbf{shredding} has a broad sense in the context of guitar playing. Generally, it refers to a particular playing style that incorporates advanced techniques and fast playing. It originates from the field of rock music, but many styles use the term shredding to refer to a particular quick melody played by a guitarist. In my application, I use the term \textbf{Shred} to refer to a video of someone playing a short guitar melody that uses some fancy technique. A tradition in the guitar playing community is to show off their "shredding" skills in videos on the Internet, like for instance Youtube or MySpace. Another fascinating scenario is the so-called guitar battle, in which one guitarist would battle against another on stage in front of an audience. In this scene two or more guitarists would take turns in playing shreds against each other, with the intent that each shred is more interesting then the one performed before. I have adopted these scenarios in the web application. Here, I use the terms \textbf{Shredder}, meaning a user on the web site. Users can register a (Shredder) profile on the page, where they upload some information about themselves, for instance where they come from, what sorts of musical instruments they own, or what types of music they like.  A shredder has a shred-level that represents his shredding skills.  Shredders are connected to each other as \textbf{Fans} and \textbf{Fanees}; a shredder A can be a fan of shredder B, in which case shredder A is a fanee of shredder B. This creates a graph of interconnected users, which is one of the most important requirements for social networking apps.  A \emph{Shred} is the name of a video that a Shredder uploads, and \emph{Battle} is a competition between two Shredders. Everyone can give a rating to the shreds that are uploaded, and Shredders are able to comment on a particular Shred. A Shred will also have a set of \textbf{Tags} \footnote{Tags is a widely adopted term in the world of Web 2.0; many web 2.0 applications uses tags to classify things like blogs and images} assigned to it. These can be attributes like "riff", "solo", "song-cover", "rock", "classical", "sweeping", "tapping", "scale",  "whammy-tricks" etc. Shredders achieve experience points when someone rates their shreds, either in a battle, or a normal shred upload. Experience points helps the shredder gain \textbf{level}, which is a measure of how skilled the player is. A major goal for the application is to create good personal shred-recommendations for each Shredder, so that they can discover new Shredders and widen their fan graph. The recommendations are created based on the Shredder's profile, such that each Shredder get recommendations for other, similar Shredders.
		
\section{User stories}
Given below is the set of user stories that has been implemented for the application. Each user story is outlined together with the URL that contains the user story. 
\begin{description}
		
\item[The front page, www.shredhub.com] \hfill \\
The user is first met with a front page as seen in figure \vref{fig:frontpage}. Here the user can either register as a new Shredder, or log in with a username and password. The user will not be able to access any of the other services in the app before he is logged in. The page also displays a set of the most popular Shred videos.
\begin{figure}
 \begin{center}
\fbox{\includegraphics[width=15cm] {images/frontpage.png}}
% Kommandoen \fbox tegner en ramme.
\end{center}
\caption{The front page at Shredhub}\label{fig:frontpage}
\end{figure}

\item [The shred pool, www.shredhub.com/theshredpool] \hfill \\
This is the first page the Shredder meets when he logs in. It is the main shred-arena, which contains multiple rows of Shreds made by other Shredders. Considering it is sort of a central page in the app, it is important that the page renders very quickly. The page contains the following rows of Shreds:
\begin{enumerate}
\item{} Newest Shreds made in the app
\item{} Shred-news:
	\begin{enumerate}
		\item{} Newest Shreds made by fanees
		\item{} Newest Shreds by fanees made in a Battle
		\item{} Newly created Battles by fanees
		\item{} New recommended Shredders to connect to
	\end{enumerate}
\item{} Shreds with particular high rating
\item{} Shreds from Shredders that might be of interest
\item{} Shreds based on tags the User enters
\end{enumerate}

Every row contains a collection of 3-5 Shreds, except the final row which contains 20 Shreds. The User can click the next button in a row, which results in a new row of Shreds. This should also happen very fast and responsive. The Shred-newst section is a set of Shreds and Shredders especially picked out to fit the User's profile, that is, new content made by his fanees, and recommendations for new Shredders. The recommendations are made to motivate the users to expand their network off Shredder-connections, which is an elementary part of typical Web 2.0 applications. 

The shredder can also create and upload a new Shred by clicking "Upload shred". If this button is clicked the Shredder will be asked to pick a movie from his computer (or smart phone/tablet), add some relevant tags for the Shred, a description and a name. Then the Shred will be saved in the database, and immediately be available to every other Shredder in the application. This screen is showed in figure \vref{fig:theshredpool}. If the Shredder clicks on a particular Shred one of the rows, a new window pops up displaying the Shred video, a description for the Shred, a name, and the set of tags created for the Shred. In addition there is a list of comments that has been made on the Shred and a number representing the computed rating for the Shred. The logged in user can choose to add a comment on the Shred, and give the Shred a rating value. Its very important that when the User adds a rating or enters a comment, the result of this is displayed very fast in the Shred window. If a rating is added, the new rating is added to the Shred, and the Shredder who made the Shred will earn more experience points. 
		
\begin{figure}
  \begin{center}
\fbox{\includegraphics[width=15cm] {images/shredpool1.png}}
% Kommandoen \fbox tegner en ramme.
\end{center}
\caption{The Shredpool (top)}\label{fig:theshredpool}
\end{figure}

\begin{figure}
  \begin{center}
\fbox{\includegraphics[width=15cm] {images/shredpool2.png}}
% Kommandoen \fbox tegner en ramme.
\end{center}
\caption{The Shredpool (bottom)}\label{fig:theshredpool}
\end{figure}

\item [Shredders, www.shredhub.com/shredders] \hfill \\
This is an overview of all the the Shredders that are using the app. Considering that the amount of Shredders on the page might be very big, the list is paginated, meaning a fixed number (20 in this case) is displayed at a time, and the Shredder can click next to iterate to the next page of Shredders. Shredders can also search for other Shredders by name. The purpose of this page is to encourage Shredders to meet new Shredders so that their fan graphs can be extended. The currently logged in Shredder can click "become fane" to become a fan of a Shredder that is in this list, or click on one of the Shredders to access his public profile page. This page can be seen in figure \vref {fig:shredders}. 
		
		
\item [Shredder, www.shredhub.com/shredder/<id>] \hfill \\
This is a page that displays the details for a given Shredder, that has the unique id found in the URL. A list of Shreds that the current shredder has published is displayed in a list view, together with a list of his fans. The logged in user may choose to challenge this Shredder for a battle. The page is customized to show the relationship the currently logged in shredder has with this Shredder. This might be that they already are in a battle, or if a battle request is sent to this shredder, if they are fans of each other already, and other similar relationships (\vref{fig:shredder}).
		
\begin{figure}
 \begin{center}
\fbox{\includegraphics[width=15cm] {images/shredders.png}}
% Kommandoen \fbox tegner en ramme.
\end{center}
\caption{The list of Shredders}\label{fig:shredders}
\end{figure}

\begin{figure}
  \begin{center}
\fbox{\includegraphics[width=15cm] {images/shredder.png}}
% Kommandoen \fbox tegner en ramme.
\end{center}
\caption{A shredder's profile page}\label{fig:shredder}
\end{figure}
		
\item [Battle] \hfill \\
This page displays a battle between two shredders. If the currently logged in user is one of the battlers, the user is able to upload a shred for the battle. In a battle I distinguish between the battler who initiates the battle, and the battlee, being the one who is challenged. I have not added an image of a battle, because it won't be discussed in much detail in this thesis.

\item{Shred} \hfill \\
A pop-up window displays a particular Shred made by a Shredder. Users can add a rating to the Shred, and add comments for it. The shred can be accessed from multiple different pages in the app. An example image is given in \vref{fig:shred}.

\item{Upload a Shred} \hfill \\
For uploading Shreds, a simple pop-up window is displayed so the User can add a Video, a description, and a set of tags. This window can only be accessed inside the Shredpool. This can be seen in \vref{fig:add shred}

\begin{figure}
  \begin{center}
\fbox{\includegraphics[width=15cm] {images/shred.png}}
% Kommandoen \fbox tegner en ramme.
\end{center}
\caption{A pop-up window displaying a Shred}\label{fig:shred}
\end{figure}


\begin{figure}
  \begin{center}
\fbox{\includegraphics[width=15cm] {images/addshred.png}}
% Kommandoen \fbox tegner en ramme.
\end{center}
\caption{A pop-up that lets the User add a new Shred}\label{fig:addshred}
\end{figure}

\end{description}
		
